\newchap{Preliminaries}{Preliminaries: \\
    Set theory \& categroies}{}


\section{Naive set theory}


\subsection{集合的运算}

\begin{defi}{集合的运算}
    \begin{itemize}
        \item \(\cup:\) union;
        \item \(\cap:\) intersection;
        \item \(\backslash:\) difference;
        \item \(\amalg :\) disjoint union;
        \item \(\times :\) set product;
    \end{itemize}
\end{defi}

\subsection{disjoint union}


\begin{defi}{Disjoint union}
    \(S \amalg T\):得到 \(S\) 与 \(T\) 的拷贝 \(S'\) 与 \(T'\),且 \(S' \cap T' = \varnothing\),则 \(S' \cup T' = S\amalg T\)。
    其中一种依赖于 set product 的实现:
    \[
        \begin{cases}
            S' \coloneqq \left\{ 0 \right\} \times S, \\
            T' \coloneqq \left\{ 1 \right\} \times T.
        \end{cases}
    \]
\end{defi}

\subsection{set product}

\begin{defi}{Set product}
    \[
        S \times T\coloneqq \left\{ \left\{ \left\{ s \right\} ,\left\{ s,t \right\}  \right\} : s\in S\land t\in T \right\}
        .\]

    将 \(\left\{ \left\{ s \right\} ,\left\{ s,t \right\}  \right\}\) 写作 \((s,t)\),称为 pair。
\end{defi}
\subsection{等价关系}


\begin{defi}{等价关系}
    若 \(\symcal{R} \) 是二元关系,则 \(a,b\) 满足关系 \(\symcal{R} \) 写为:
    \[
        a\mathop{\symcal{R}}b
        .\]

    若关系 \(\sim\) 定义在集合 \(S\)上满足:
    \begin{itemize}
        \item reflexivity: \((\forall a\in S) a \sim a\).
        \item symmetry: \((\forall a\in S)(\forall b\in S) a\sim b \implies b \sim a\).
        \item transitivity: \((\forall a\in S)(\forall b\in S)(\forall c\in S) a\sim b \land b\sim c \implies a\sim c\).
    \end{itemize}
    则称 \(\sim\) 是在集合 \(S\) 上的等价关系。
\end{defi}


\subsection{分划与等价类 (partition \& equivalence class)}

\begin{defi}{分划与等价类}
    \begin{itemize}
        \item \textbf{分划}是一个集合的集合,满足:
              \[
                  \begin{cases}
                      (\forall a\in P) (\forall b \in P) a \cap b = \varnothing, \\
                      \bigcup_{a \in P} a = S.
                  \end{cases}
              \]
              则称 \(P\) 是 \(S\) 的分划。
        \item \textbf{等价类}:
              \[
                  [a]_{\sim} \coloneqq  \left\{ x \in S : x \in a \right\}
                  .\]
              称此为在 \(S\) 上 \(a\) 的等价类,由于等价类两两不交,且具有自反性,则 \(S\) 上某等价关系得到的所有等价类组成的集合是 \(S\) 的分划 \(\symcal P_{\sim}\)。
    \end{itemize}
\end{defi}


\subsection{集合商 (set quotient)}
\begin{defi}{集合商}
    集合 \(S\) 与等价关系 \(\sim\) 的商定义为:
    \[
        S /\mathord \sim \coloneqq\symcal{P} _{\sim}
        .\]
\end{defi}

即 \(a,b\in S\) 等价 \(\iff\) 商到同一个元素。

\begin{quote}{一个集合商的例子}
    定义 \(\symbb{Z} \) 上的等价关系 \(\sim\) : \(a\sim b \iff \frac{a - b}{2} \in\symbb{Z}\),则:
    \[
        \symbb{Z} /\mathord \sim = \left\{ [0]_{\sim} , [1]_{\sim} \right\}
        .\]
\end{quote}


\section{Functions between sets}


\subsection{函数}

\begin{defi}{函数}
    \begin{itemize}
        \item 函数的 Graph:
              \[
                  \Gamma _f \coloneqq \left\{ (a,b) \in A \times B : b = f(a) \right\}
                  .\]
              且满足 \((\forall a\in A)(\exists !b\in B) (a,b) \in \Gamma_f \),即 \((\forall a\in A)(\exists !b\in B) f(a) = b\)。
        \item 函数的图的表示:

              \[
                  \begin{cases}
                      A \xrightarrow{f} B, \\
                      a \mapsto f(a).
                  \end{cases}
              \]
    \end{itemize}
\end{defi}

\subsection{Indentity function(id)}

在集合 \(A\) 上有:
\[
    \id_A : A\to A , (\forall a\in A)\id_A(a) = a
    .\]
\subsection{函数的 image}

若 \(S\subset A, f : A \to B\),则:

\[
    f(S) \coloneqq \left\{ b\in B : (\exists a \in S) f(a) = b  \right\}
    .\]

则 \(f(A)\) 就是函数的image,记作 \(\operatorname{im}f\)

\subsection{函数的 restriction}

记 \(S\subset A\),则:
\[
    f|_S : S \to A, (\forall s\in S) f|_S(s) = f(s)
    .\]

\subsection{函数的复合 (composition)}

\begin{itemize}
    \item 若 \(f:A\to B, g: B\to C\),则 \(g\circ f: A\to C, (\forall a\in A) g\circ f(a) \coloneqq g\left( f(a) \right) \)。
\end{itemize}
\[
    \begin{tikzcd}
        A \arrow[rd, "g \circ f"]\arrow[r, "f"] & B\arrow[d, "g"] \\
        {}                                      & C
    \end{tikzcd}
\]
{\slshape 此时称图是交换(commutative)的,因为图描述的所有从 \(A\) 到 \(C\) 的通路都会送 \(A\) 中的任意一个元素到相同的结果。 }

函数的复合满足结合律:
\[
    \begin{tikzcd}
        A\arrow[r, "f"]\arrow[rr, "g\circ f", bend left = -40] & B\arrow[r, "g"]\arrow[rr, "g\circ f", bend left = 40] & C\arrow[r, "h"] & D
    \end{tikzcd}
    .\]
即 \(h \circ (g \circ f) = (h\circ g) \circ f\)。

\subsection{单射、全射、双射(injections, surjections, bijections)}


\begin{defi}{单射 (Injections, \textbf{Inj})}
    \(f: A\to B\) 是单的若:
    \[
        (\forall a'\in A)(\forall a''\in A) a' \neq a'' \implies f\left( a' \right) \neq f(a'')
        .\]
    实际上就是 \((\forall a'\in A)(\forall a''\in A) f\left( a' \right) = f\left( a'' \right) \implies a'=a''\)。
    一般用箭头 \(f: A ↪ B \) 表示。
\end{defi}

\begin{defi}{全射 (Surjections, \textbf{Surj})}
    \(f : A\to B\) 是全的若:
    \[
        (\forall b\in B)(\exists a\in A) f(a) = b
        .\]
    此时 \(\operatorname{im}f = B\)。
    一般用箭头 \(f: A ↠ B\) 表示。
\end{defi}

\begin{defi}{双射 (Bijections, \textbf{Bij})}
    \(f\) 是双的当且仅当 \(f\) 又单又全,一般用箭头 \(f: A\stackrel{\sim}{\to }B\) 表示。
    \begin{itemize}
        \item 若 \(\exists f: A \stackrel{\sim}{\to } B\),则记此时 \(A\cong B\),若其中一个集合元素数量有限,则另一个也有限且两个集合元素数量相等。
        \item 集合 \(A\) 中的元素数量写作 \(|A|\);幂集写作 \(2^A\)。
    \end{itemize}
\end{defi}

\subsection{单射、全射、双射的性质}
\begin{itemize}
    \item 双射有逆 (inverse):
          \begin{theo}{双射有逆}{}
              定义函数 \(f: A\stackrel{\sim}{\to }B\),定义 \(g: B \to 2^A, (\forall b\in B)g(b) = \left\{ a : f(a) = b \right\} \),则由于 \(f\) 是单的,则 \((\forall a',a''\in A)f(a') = f(a'') \implies a' = a'\),故 \((\forall b\in B ) |g(b)| = 1\)。故可以定义 \(g': B \to A, (\forall b\in B) g(b) = a, \text{ \underline{st} } f(a) = b\),且是良定义的。

              此时 \(g'\circ f = \id_A, f\circ g' = \id_B\),此称 \(g'\) 为 \(f\) 的逆,记为 \(f^{-1} \)。
          \end{theo}

          双射的逆唯一:
          \begin{proof}{双射的逆唯一}
              定义 \(f : A \stackrel{\sim}{\to} B\) 的逆 \(g,g'\),由于 \(f\circ\id_A = f = \id_B\circ f\),因此:
              \[
                  g = g\circ \id_B = g\circ (f \circ g') = (g \circ f) \circ g' = \id_A \circ g' = g'
                  .\]
              故唯一。
          \end{proof}
    \item 左逆与右逆 (Linv \& Rinv)
          若 \(f: A\to B, g\circ f = \id_A\),则称 \(g\) 是 \(f\) 的左逆,同理有右逆。

          如果 \(A \neq \varnothing, f:A\to B\):
          \begin{itemize}
              \item \(f\) 有 Linv \(\iff f\) 是单的。
                    \begin{proof}{}
                        \begin{enumerate}
                            \item (\(\implies \)) 若 \(f\) 有左逆,设为 \(f^{-1} \),则:
                                  \[
                                      (\forall a,b\in A\land a\neq b)f^{-1} (f(a)) = \id_A(a) = a \neq b = f^{-1} (f(b))
                                      .\]
                                  若 \((\exists a,b\in A)f(a) = f(b)\),则与上式矛盾,故 \(f\) 是单的。
                            \item (\(\impliedby \)) 若 \(f\) 是单的,有双射有逆那部分的讨论知道 \((\forall b\in\operatorname{im}f)\exists !a \in A \text{ \underline{st} } f(a) = b\),故定义:
                                  \[
                                      g(b)\coloneqq \begin{cases}
                                          a, & \text{if }(\exists a\in A)b = f(a),    \\
                                          S, & \text{if } b \notin \operatorname{im}f.
                                      \end{cases}
                                  \]
                                  则 \(g\) 满足 \(g\circ f = \id_A\)。
                        \end{enumerate}
                    \end{proof}
              \item \(f\) 有 Rinv \(\iff f\) 是全的。
                    \begin{proof}{}
                        只证明 \((\impliedby )\) 的部分,由于 \(f\) 是全的,定义:
                        \[
                            g:B\to 2^A, b\mapsto  \left\{ a\in A : f(a) = b\right\}
                            .\]
                        则 \((\forall b\in B)g(b) \neq \varnothing\),定义:
                        \[
                            h:B\to A, b \mapsto a \text{ \underline{st} } a   \in g(b)
                            .\]
                        这样定义的 \(h\) 可能有很多种,但都满足其是 \(f\) 的右逆:
                        \[
                            (\forall b\in B)h(b) \in g(b) \implies f\circ h(b) = \id_B
                            .\]
                    \end{proof}
              \item 若\(f\)同时有左逆和右逆,则两个逆相同。
          \end{itemize}
\end{itemize}
\begin{quote}{一些单射、全射的例子}
    \begin{itemize}
        \item 投影
              \[
                  \begin{tikzcd}
                      {} & A\times B\arrow[rd, "\pi_B", two heads]\arrow[ld, "\pi_A"', two heads] & {} \\
                      A  & {}                                                                     & B  \\
                  \end{tikzcd}
              \]
              其中 \(\pi \) 是投影 (projection) 映射:
              \[
                  \begin{aligned}
                      \pi _A((a,b)) \coloneqq a , \\
                      \pi _B((a,b))\coloneqq b.
                  \end{aligned}
                  .\]
              是全射。

        \item 与不交并的映射
              \[
                  \begin{tikzcd}
                      A\arrow[rd, "f_B",hook] & {}         & B\arrow[ld, "f_A"', hook'] \\
                      {}                      & A\amalg  B & {}                         \\
                  \end{tikzcd}
              \]
              若将 \(A\amalg B\) 表示为 \(A'\cup B'\),其中 \(A'\overset{F_A}{\cong} A, B'\overset{F_B}\cong B\),则 \((\forall a\in A)f_A(a) \coloneqq F_A(a) \in A\amalg B\)。
        \item 商
              \[
                  \begin{tikzcd}
                      A\arrow[r, "f", two heads] & A /\mathord \sim
                  \end{tikzcd}
              \]
        \item 函数的标准分解 (Canonical decomposition)
              对函数 \(f:A\to B\),在 \(A\) 上建立等价关系:
              \[
                  a\sim b \iff f(a) = f(b)
                  .\]
              则函数可以分解为:
              \[
                  \begin{tikzcd}
                      A\arrow[r, two heads]\arrow[rrr, bend left = 40, "f"] & A /\mathord \sim \arrow[r, "\sim", "\tilde f"'] & \operatorname{im}f\arrow[r, hook] & B
                  \end{tikzcd}
              \]
              其中 \(\tilde f:A / \mathord\sim \to \operatorname{im} f, \tilde f([a]_{\sim}) \coloneqq f(a)  \),不难验证这是良定义的,现在证明 \(\tilde f\) 是双射:
              \begin{proof}{}
                  只需证明 \(f\) 既是单射也是全射就可以了:
                  \begin{enumerate}
                      \item \textbf{inj}:
                            \[
                                \tilde f\left( [a]_{\sim} \right)  = \tilde f\left( [b]_{\sim} \right) \implies f(a) = f(b) \implies a\sim b \implies [a]_{\sim} = [b]_{\sim}
                                .\]
                      \item \textbf{surj}:
                            \[
                                (\forall b\in \operatorname{im}f)\exists a\in A \text{ \underline{st} }f(a) = b \implies \tilde f\left( [a]_{\sim} \right) = b
                                .\]
                  \end{enumerate}
              \end{proof}
    \end{itemize}
\end{quote}


\section{范畴 (Categories)}



一个范畴 \(\symsf C\)包括:
\begin{itemize}
    \item 一个类 \(\Obj(\symsf C)\),包括了对象 (object)。
    \item 对任意两个对象 \(A,B\) 存在一个集合记为 \(\Hom_{\symsf C}(A,B)\) 包含了从 \(A\) 到 \(B\) 的全部态射 (morphisms),态射和 \(\Hom\) 满足以下特点:
          \begin{itemize}
              \item \textbf{幺元的存在性}

                    \(\forall A\in \symsf C,\exists 1_A \in \Hom_{\symsf C}(A,A) \eqcolon \End_{\symsf C}(A)\),称为 \(A\) 的 indentity。
              \item \textbf{态射复合的存在性}

                    若 \(\exists f\in \Hom_{\symsf C}(A,B), g\in \Hom_{\symsf C}(A,B)\),则存在 \(f,g\) 决定的态射 \(gf\in \Hom_{\symsf C}(A,C)\),由于 \(\Hom\) 是集合,因此存在函数:
                    \[
                        \Hom_{\symsf C}(A,B) \times \Hom_{\symsf C}(B,C) \to \Hom_{\symsf C}(A,C)
                        .\]
              \item \textbf{态射复合的结合性}

                    若 \(f\in \Hom_{\symsf C}(A,B), g\in \Hom_{\symsf C}(B,C), h\in \Hom_{\symsf C}(C,D)\),则
                    \[
                        (hg)f = h(gf)
                        .\]
                    这一性质导致态射图可交换。
              \item \textbf{幺元律}

                    \[
                        \forall f\in \Hom_{\symsf C}(A,B),f 1_A = 1_Bf=f
                        .\]
          \end{itemize}
\end{itemize}

\begin{quote}{一些范畴的小例子}
    \begin{enumerate}[label = \textbf{\arabic*.}]
        \item 对象为集合、态射为集合函数的范畴,记为 \(\SET\):
              \begin{itemize}
                  \item \(\Obj(\SET) \coloneqq \) 一个包含所有集合的类。
                  \item \(\Hom_{\SET}(A,B) \coloneqq B^A\)。
              \end{itemize}
        \item 一个关于二元运算的范畴:若 \(S\) 上的二元运算 \(\sim\) 满足:
              \[
                  (\forall a,b,c\in S)\begin{cases}
                      a\sim a, \\
                      a\sim b\land b\sim c\implies a\sim c .
                  \end{cases}
              \]
              则定义 \(\symsf C_{\sim}\):
              \begin{itemize}
                  \item \(\Obj\left( \symsf C_{\sim} \right) \coloneqq S\)。
                  \item \(\Hom_{\symsf C_{\sim}}(A,B)\coloneqq \begin{cases}
                            (A,B),       & \text{if }A\sim B,  \\
                            \varnothing, & \text{if }A\nsim B.
                        \end{cases}\)
              \end{itemize}
              定义复合为:
              \[
                  \circ _{\symsf C_{\sim}} :((A,B),(B,C))\mapsto (A,C)
                  .\]
              则其为一个范畴。
              \begin{itemize}
                  \item 一个特例,如果认为 \(S=\symbb{Z} \),\(\sim\) 为 \(\leqslant \),则态射图如下:
                        \[
                            \begin{tikzcd}
                                2\arrow[r]\arrow[rd] & 3\arrow[r]\arrow[d, "1_3"] & 4\arrow[r]  & 5  \\
                                {}                   & 3\arrow[r]\arrow[rru]      & 4\arrow[ru] & {}
                            \end{tikzcd}
                        \]
              \end{itemize}
        \item \textbf{由范畴诱导范畴}
              \begin{itemize}
                  \item [\scshape\bfseries Slice CAT]
                        考虑范畴 \(\symsf C\) 中的对象 \(A\),接下来构建 \(\symsf C_A\):
                        \begin{itemize}
                            \item \(\Obj(\symsf C_A) \coloneqq \)  \(\symsf C\) 中所有到 \(A\) 的态射。
                            \item \(\Hom_{\symsf C_A}(f_1,f_2) = \left\{ \sigma :f_1=f_2 \sigma  \right\} \)。
                        \end{itemize}
                        \[
                            \begin{tikzcd}
                                Z_1\arrow[d, "f_1"']\arrow[rd, "\sigma "] & {}                    \\
                                A                                         & \arrow[l,"f_2"']  Z_2 \\
                            \end{tikzcd}
                        \]
                        \(\symsf C_A\) 中态射的复合取自 \(\symsf C\) 中的态射复合。
                  \item [\scshape\bfseries CoSlice CAT]
                        同理,只不过将从到 \(A\) 变成了 \(A\) 到其他对象的态射。
                  \item [\scshape\bfseries Opp CAT]
                        \begin{itemize}
                            \[
                                \begin{cases}
                                    \Obj(\symsf C^{\symup{op}})\coloneqq \Obj(\symsf C), \\
                                    \Hom_{\symsf C^{\symup{op}}}(A,B)\coloneqq \Hom_{\symsf C}(B,A).
                                \end{cases}
                            \]
                        \end{itemize}
              \end{itemize}
    \end{enumerate}
\end{quote}


\subsection{态射们}



\begin{defi}{同构 (Isomorphisms)}
    若一个态射 \(f \in \Hom_{\symsf C}(A,B)\) 满足:
    \[
        \exists g\in \Hom_{\symsf C}(B,A) \st gf=1_A, fg=1_B
        .\]
    则 \(f\) 是一个同构,此时记 \(g\) 为 \(f^{-1} \),如果 \(f\) 有左逆、右逆,则它们必然相等(唯一)。
\end{defi}

\begin{quote}{一些关于逆相关范畴的例子}
    \begin{itemize}
        \item 一个同构都是 indentity 的例子,用 \((\symbb{Z} ,\leqslant )\)定义的范畴。
        \item 一个每个态射都是同构的例子,用 \((\symbb{Z} ,=)\) 定义的范畴。这种性质的范畴被称为广群 (Groupoids)。
    \end{itemize}
\end{quote}

\begin{defi}{自同构 (Automorphisms)}
    就是属于 \(\End\) 的同构,所有 \(A\) 的自同构组成的集合称为 \(\Aut_{\symsf C}(A)\)。
    \begin{itemize}
        \item \(f,g \in\Aut_{\symsf C}(A) \implies fg\in\Aut_{\symsf C}(A)\)。
        \item \(f\in \Aut_{\symsf C}(A) \implies f^{-1} \in\Aut_{\symsf C}(A)\)。
    \end{itemize}
    \(\Aut\) 是一个\textbf{群} (Group)。
\end{defi}

\begin{defi}{单态射 (Monomophisms, \textbf{Monic})}
    即满足左消去律的态射:
    \begin{gather*}
        \forall Z\in \Obj(\symsf C),\forall a,b\in\Hom_{\symsf C}(Z,A),f:A\to B\\
        f \textup{ is a monic} \iff \left( f\circ a=f\circ b \implies a=b \right)
    \end{gather*}
\end{defi}

\begin{defi}{全态射 (Epimorphisms, \textbf{Epic})}
    满足右消去律的态射:
    \begin{gather*}
        \forall Z\in \Obj(\symsf C),\forall a,b\in\Hom_{\symsf C}(B,Z) ,f:A\to B\\
        f \textup{ is an epic} \iff \left( a\circ f=b\circ f \implies a=b \right)
    \end{gather*}
\end{defi}


在 \(\SET\) 中,单态射和全态射就是集合之间的单射和全射。

\begin{proof}{\(\SET\)中的单/全态射是集合之间的单/全射}
    \((\impliedby) \),只需考虑单/全射的左/右逆即可:
    \[
        f\circ a=f\circ b\implies f^{-1} \circ f\circ a=f^{-1} \circ f\circ b\implies a=b
        .\]

    \((\implies)\),可以用反证法,若 \(f\) 是非单射但是是单态射,则 \(\exists a\neq b(f(a)=f(b))\),考虑态射 \(A:\left\{ * \right\} \to a,B:\left\{ * \right\} \to b\),则 \( A \neq B\land f\circ A = f\circ B\),与单态射的定义矛盾。

    类似的,若 \(f:A\to B\) 是非全射且是全态射,则 \(B\backslash {\operatorname{\symrm{im}}f}\neq \emptyset\),定义态射 \(X: B\to \left\{ 1 \right\} ,Y:B\to \left\{ 0,1 \right\}\),且:
    \[
        Y(y)\coloneqq \begin{cases}
            1,  & \text{if }y\in \operatorname{im}f ,   \\
            0 , & \text{if }y \notin\operatorname{im}f.
        \end{cases}
    \]
    则同样与全态射的定义矛盾。
\end{proof}

\textbf{注意!Iso 并不等于 Monic \(\land\) Epic!}具体例子可以见 \((\symbb{Z} ,\leqslant )\)所定义的范畴:每个 \(\Hom\) 中只有一个态射,则必然左/右可消去,但只有 \(\End\) 是同构。同时,Monic的复合是Monic,Epic同理。


\section{泛性质(Universal properties)}

泛性质与 I(nitial) / F(inal) 对象有关:

\begin{defi}{I 对象与 F 对象}
    \(A\in\Obj\left( \symsf C \right) \),则 \(A\) 是 I 的若:
    \[
        (\forall Z\in\Obj(\symsf C))\left\vert \Hom_{\symsf C}(A,Z) \right\vert =1
        .\]
    \(A\) 是 F 的若:
    \[
        (\forall Z\in\Obj(\symsf C))\left\vert \Hom_{\symsf C}(Z,A) \right\vert =1
        .\]
    若 \(I_1,I_2\) 是 \(\symsf C\) 上的 I / F 对象,则 \(I_1\cong I_2\)。
\end{defi}


\subsection{泛性质与一些例子}

泛性质长得像一个范畴的 I/F 对象,比如:

\begin{quote}{空集的泛性质是「集合之间的映射」}
    因为以集合为对象、集合映射为态射的范畴 \(\SET\) 中,空集是 I 对象。
\end{quote}

\begin{quote}{一些其他例子}
    \begin{itemize}
        \item 集合商 \(A / \mathord\sim\) 的泛性质是从集合 \(A\) 到其他映射集合的映射,满足:“等价的 \(A\)中元素有相同的像。”
              即
              \[
                  A\xrightarrow{f}Z, f\st a\sim b \implies f(a) = f(b)
                  .\]
              以此为范畴 \(\symsf C_{A,\sim}\) 的 \(\Obj\),则态射为 \(\Hom(f_1,f_2) = \left\{ \sigma :\sigma f_1=f_2 \right\} \),则考虑以下cd:
              \[
                  \begin{tikzcd}
                      A/\mathord \sim\arrow[r, "\exists !\sigma"]          & Z  \\
                      A \arrow[u, "\pi"]\arrow[ur, "f_A", bend right = 20] & {}
                  \end{tikzcd}
              \]
              其中 \(\pi \) 已给定(为商投影映射),则 \(A/\mathord\sim\) 是这个范畴的 I 对象。
              \begin{proof}{}
                  \(\forall a\in A\) 都有 \(\sigma \pi (a) = f_A(a)\),即 \(\sigma \left( [a]_{\sim} \right) = f_A(a) \),此就相当于定义了 \(\sigma\)(保证唯一),易证 \(\sigma\) 是良定义的。
              \end{proof}
              同时,\(\operatorname{im}f\) 也是其 I 对象:
              \[
                  \begin{tikzcd}
                      \operatorname{im}f \sim\arrow[r, "\exists !\sigma'"] & Z  \\
                      A \arrow[u, "f"]\arrow[ur, "f_A", bend right = 20]   & {}
                  \end{tikzcd}
              \]
              故由 I 对象的特点有 \(\operatorname{im}f \cong A / \mathord\sim\)。
        \item 集合的积
              集合 \(A,B\) 的积的泛性质是一个集合到 \(A\) 和 \(B\) 的两个映射。

              给出三元组 \((Z,f_A,f_B)\),此为 \(\symsf C\) 的 \(\Obj\)。则其态射为:
              \[
                  \Hom\left( (Z_1,f_A,f_B),(Z_2,g_A,g_B) \right) \coloneqq \left\{ \sigma:g_A\sigma = f_A\land g_B\sigma = f_B \right\}
                  .\]
              \(A\times B\)是其 F 对象:
              \[
                  \begin{tikzcd}
                      {}                                                                & A  & {}                                                  \\
                      Z\arrow[ur, "f_A"]\arrow[dr, "f_B"']\arrow[rr, "\exists !\sigma"] & {} & A\times B \arrow[ul, "\pi _A'"]\arrow[ld, "\pi _B"] \\
                      {}                                                                & B  & {}
                  \end{tikzcd}
              \]
              对 \(\forall z\in Z\) ,都有:
              \[
                  \begin{cases}
                      \pi _A\sigma (z) = f_A(z), \\
                      \pi _B\sigma(z) = f_B(z).
                  \end{cases}
              \]
              故 \(\sigma : z\mapsto \left( f_A(z), f_B(z) \right) \),唯一。

              定义 \(A \times B\) 中的积 (product) 为 \(\symsf C_{A,B}\) 那个的 F 对象(若存在)。
              \begin{itemize}
                  \item 另一个例子,在 \(\symbb{Z}, \leqslant \) 定义的范畴中, \(A\times B\coloneqq \min(A,B)\)。
              \end{itemize}
        \item 余积 (Coproduct)
              定义 \(A,B\)余积 \(A\amalg B\)为 \(\symsf C^{A,B}\) 中的 I 对象,则 \(\SET\) 中的余积为两个集合的不交并。

              \begin{proof}{}
                  \[
                      \begin{tikzcd}
                          {}                                  & A\amalg B \arrow[dd, "\exists !\sigma"] & {}                                  \\
                          A\arrow[ru, "I_A"]\arrow[rd, "f_A"] & {}                                      & B\arrow[ul, "I_B"]\arrow[ld, "f_B"] \\
                          {}                                  & Z                                       & {}
                      \end{tikzcd}
                  \]
                  如图所示,考虑 \(A \amalg B\) 的一种实现:
                  \[
                      A\cong A',B\cong B',A'\cap B'=\varnothing,A'\cup B'\cong A\amalg B
                      .\]
                  则,\(\begin{cases}
                      \forall a\in A, \sigma I_A(a) = f_A(a) \\
                      \forall b\in B, \sigma I_B(b) = f_B(b)
                  \end{cases}\),故:
                  \[
                      \sigma : a\mapsto \begin{cases}
                          f_A\left( I_A^{-1} |_{A'} (a) \right) , & \text{if } a \in A', \\
                          f_B\left( I_B^{-1} |_{B'} (b) \right) , & \text{if } a \in B'.
                      \end{cases}
                  \]
              \end{proof}
    \end{itemize}
\end{quote}

\begin{itemize}
    \item 纤维积 (Fiber product)

          \begin{defi}{纤维积}
              首先定义范畴 \(\symsf C_{\alpha, \beta }\):
              \[
                  \begin{tikzcd}
                      {}                               & A\arrow[rd, "\alpha "] & {} \\
                      Z\arrow[ru, "f"]\arrow[rd, "g"'] & {}                     & C  \\
                      {}                               & B\arrow[ru, "\beta"' ] & {}
                  \end{tikzcd}
              \]
              其 \(\Obj\) 是如上三元组 \((Z,f,g)\) 满足 \(\alpha f=\beta g\)。其态射为
              \[
                  \Hom((Z_1,f_1,g_1),(Z_2,f_2,g_2)) \coloneqq \left\{ \sigma : f_1=f_2\sigma \land  g_1=g_2\sigma\right\}
                  .\]
              看起来和 \(\symsf C_{A,B}\) 很像,只不过交换图要求更高了。定义 \(A,B\) 的纤维余积 \(A \times _C B\)为此范畴的 F 对象。
          \end{defi}

          \(\SET\) 上的纤维积可以如下定义:

          \[
              \begin{tikzcd}
                  {}                                                     & A\arrow[rd, "\alpha "] & {} \\
                  A \times _C B\arrow[ru, "\pi _A"]\arrow[rd, "\pi _B"'] & {}                     & C  \\
                  {}                                                     & B\arrow[ru, "\beta" '] & {}
              \end{tikzcd}
          \]
          不妨设 \(A\times _CB\subset A\times B\),由于态射图要交换,即 \(\alpha \pi _A=\beta \pi _B\),故 \(A\times _CB\coloneqq \left\{ (x,y) : \alpha (x)  = \beta (y)\right\} \)。现在来证明 \(A\times _CB\) 是终对象:
          \begin{proof}{}
              对于 \(\forall Z\),若存在 \(Z\) 到 \(A,B\) 的映射 \(f_A,f_B\)满足 \(\alpha f_A=\beta f_B\),则 \(\exists !\psi \) 满足 \(f_A = \pi _A \psi \land f_B = \pi _B \psi  \),不妨设 \(\psi \) 将 \(z\) 映射到 \((\psi _A(z),\psi _B(z))\)。则易得 \(\psi _A=f_A,\psi _B=f_B\),因此 \(\psi \) 是存在且唯一的。
          \end{proof}

    \item 纤维余积 (Fiber coproudct)

          \begin{defi}{纤维余积}

              定义范畴 \(\symsf C ^{\alpha,\beta }\):
              \[
                  \begin{tikzcd}
                      {} & A\arrow[ld, "f_A"]   & {}                                          \\
                      Z  & {}                   & C\arrow[lu, "\alpha "]\arrow[ld, "\beta "'] \\
                      {} & B\arrow[lu, "f_B" '] & {}
                  \end{tikzcd}
              \]

              以上是 \(\symsf C^{\alpha ,\beta }\) 的 \(\Obj\),其态射为定义为:
              \[
                  \Hom((Z_1,f_1,g_1),(Z_2,f_2,g_2)) \coloneqq \left\{ \sigma : \sigma f_1=f_2 \land  \sigma g_1=g_2\right\}
                  .\]
              则纤维余积是这个态射的 I 对象。
          \end{defi}

          以下是 \(\SET\) 上的纤维余积:


          重点是要解决态射图的“交换性质”,即 \((\forall z\in C )(f_A \alpha (z) = f_B \beta (z))\),同时, \(I_A\) 也会将同一个元素映射到同一个元素,故设定价关系:
          \[
              a\sim_A b \iff \alpha (a) = \alpha (b)
              .\]
          故 \([a]_{\sim_A}\subset C\) 中的所有元素都会被映射到 \(A\amalg _C B\) 中的同一个元素,若 \([a]_{\sim_A}\cap[b]_{\sim_B}\neq \varnothing\),则这两个等价类中的元素也都会映射到 \(A\amalg _CB\)中的同一个元素,故考虑等价关系:
          \[
              \begin{aligned}
                  [a]_{\sim_A}\sim_C[b]_{\sim_B} & \iff [a]_{\sim_A}\cap[b]_{\sim_B}\neq \varnothing, \\
                  [a]_{\sim_A}\sim_C[b]_{\sim_A} & \iff a=b.
              \end{aligned}
          \]
          故考虑商集:
          \[
              (C/\mathord\sim_A \amalg C/\mathord\sim_B) / \mathord \sim_C
              .\]
          则满足交换性质。另若 \(a\notin \operatorname{im}\alpha\),则可映射到自身(的等价类),因此可以认为:
          \[
              A\amalg _CB\cong (C/\mathord\sim_A \amalg C/\mathord\sim_B) / \mathord \sim_C \cup((A\backslash\! \operatorname{im}\alpha )\amalg (B\backslash\! \operatorname{im}\beta  ))
              .\]
          另外一个不太明显的想法是直接在 \(A\amalg B\)上直接商:

          考虑等价关系 \(\sim\),满足 \(A\amalg B\) 被其商掉后的商集满足态射图的交换。也就是说若 \(\alpha ^{-1}(z_1) \cap \beta ^{-1} (z_2) \neq \varnothing\),则这样 \(z_1\sim z_2\),在商后会将 \(C\) 中的一大把元素映射到 \(A\amalg B\) 中的一个元素。

          \[
              \sim\coloneqq  \begin{cases}
                  (z_1,A)\sim(z_2,B)\iff \alpha ^{-1}(z_1) \cap \beta ^{-1} (z_2) \neq \varnothing, \\
                  (z_1,A)\sim(z_2,A)\iff z_1=z_2.
              \end{cases}
          \]

          则 \(A\amalg _CB\coloneqq A\amalg B /\mathord \sim\)。

          \[
              \begin{tikzcd}
                  C\arrow[rr, "\alpha "]\arrow[dd, "\beta "]                               & {}                       & A\arrow[ld, "J_A"]\arrow[dd, "I_A"]\arrow[rddd, "f_A", bend left = 20] & {} \\
                  {}                                                                       & A\amalg B\arrow[rd, "q"] & {}                                                                     & {} \\
                  B\arrow[ru, "J_B"]\arrow[rr, "I_B"]\arrow[rrrd, "f_B",  bend left = -20] & {}                       & {}  A\amalg _CB\arrow[rd, "\psi "]                                     & {} \\
                  {}                                                                       & {}                       & {}                                                                     & Z
              \end{tikzcd}
          \]
          接下来我们知道对于 \(\forall c\in C\),都有 \(\alpha ^{-1} (\alpha (c))\cap\beta^{-1} (\beta(c))\neq \varnothing\),也即 \(J_A\alpha (c)\sim J_B \beta (c)\),因此在集合商之后有:
          \[
              qJ_A\alpha (c)= qJ_B \beta (c) \implies I_A\alpha = I_B \beta
              .\]
          其中,\(J_A,J_B\) 是不满足态射图的交换的,但最终 \(I_A \)和 \(I_B\) 满足。接下来证明这是个 I 对象:
          \begin{proof}{}
              设 \(\psi :A\amalg _CB\to (Z,f_A,f_B)\),则有 \(\psi I_A = f_A, \psi I_B = f_B\),故 \((x,A)\mapsto [(x,A)]_{\sim} \mapsto \psi \left(  [(x,A)]_{\sim}\right) =f_A(x)\),故:
              \[
                  \psi \left( [x,?]_{\sim} \right) = f_?(x) ,?\in\left\{ A,B \right\}
                  .\]
              由于如果 \([x,A]\sim[y,B] \implies  \alpha ^{-1}(x) \cap \beta ^{-1} (y) \neq \varnothing \implies\forall m_1,m_2\in\alpha ^{-1}(x) \cup \beta ^{-1} (y)\),则 \(m_1,m_2\)在 \(Z\) 中的像都相同(由于态射图的交换性质),因此 \(A\amalg _CB\)的确为 I 对象。

              如果是 \((C/\mathord\sim_A \amalg C/\mathord\sim_B) / \mathord \sim_C \cup((A\backslash\! \operatorname{im}\alpha )\amalg (B\backslash\! \operatorname{im}\beta  ))\)形状的纤维余积,则考虑
              \[
                  I_A : a\mapsto \begin{cases}
                      \left[ [\alpha ^{-1} (a)]_{\sim_A} \right] _{\sim_C}, & \text{if }a\in \operatorname{im}\alpha , \\
                      a,                                                    & \text{otherwise}.
                  \end{cases}
              \]
              \(I_B\) 同理。则:
              \[
                  \psi :x\mapsto \begin{cases}
                      [[x]_{\sim_?}]_{\sim_C}, & \text{if }x \in C/\mathord\sim_A \amalg C/\mathord\sim_B,                                                \\
                      f_?(x),                  & \text{if } x\in (A\backslash\! \operatorname{im}\alpha )\amalg (B\backslash\! \operatorname{im}\beta  ).
                  \end{cases}
              \]
          \end{proof}

\end{itemize}

