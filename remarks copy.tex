\documentclass[twosides]{memoir}
\usepackage{tikz-cd}
\usepackage{enumitem}
\usepackage{amsmath}
\usepackage{tabularray}
\usepackage{unicode-math}
\usepackage{ctex}
\usepackage{calc}
\usepackage{lipsum}
\usepackage{color,calc}
\usepackage{ninecolors}
\usepackage{tikz}
\usepackage{wallpaper}
\usepackage{titlesec}
\usepackage{tcolorbox}
\usepackage{fontspec}
\usepackage{mathtools}
\usepackage{mleftright}
\usepackage{amsthm}
\usepackage{pxrubrica}
\usepackage{hyperref}
\usepackage{multicol}
\usetikzlibrary{patterns}
\usetikzlibrary{arrows.meta}
\tcbuselibrary{skins}
\tcbuselibrary{breakable}
\tcbuselibrary{theorems}
\tcbuselibrary{most}
\let\left\mleft
\let\right\mright
\setcounter{tocdepth}{5}
\setsansfont{Noto Sans}

\makeatletter
\newCJKfontfamily\NotoSerifCJKJP{Noto Serif CJK JP}
\renewcommand{\cftchapteraftersnumb}{\hfill}
\renewcommand{\cftchapterfont}{\sffamily\large\bfseries}
\renewcommand{\cftchaptername}{{\scshape\large\chaptername}~}
\renewcommand{\cftchapterafterpnum}{\hspace*{-0.45ex}\par\vspace*{1em}}
\renewcommand{\cftchapterformatpnum}{\sffamily\large\bfseries\color{white}\colorbox{ChapBlue}}
\renewcommand*{\chapternumberline}[1]{\hspace*{-3em}\colorbox{ChapBlue}{\color{white}\cftchaptername #1}\cftchapteraftersnum\hfill\itshape\scshape\LARGE}

\renewcommand*{\cftsectionleader}{~\color{ChapBlue}\sffamily\small}
\renewcommand{\cftsectionfont}{\sffamily\scshape\small\bfseries}
\renewcommand{\cftsectionname}{{\small\bfseries\color{ChapBlue} \S}\space}
\renewcommand{\cftsectionafterpnum}{\cftparfillskip}
\renewcommand*{\cftsectionformatpnum}[1]{\sffamily\small{\color{ChapBlue}\titlerule*[0.2pc]{\(\cdot\)}}~~\textit{#1}}

\renewcommand*{\cftsubsectionleader}{~\color{ChapBlue!20!white}\sffamily \(\longmapsto\kern-0.8em{\color{white}\special{pdf:literal direct 2 Tr 0.3 w }∎\special{pdf:literal direct 0 Tr 0 w }}\)\kern-1.5em\titlerule*[0.1pc]{\(-\)}\(\!\!\!\longrightarrow\)}
\renewcommand{\cftsubsectionfont}{\footnotesize\slshape}
\renewcommand*{\cftsubsectionformatpnum}{~~\itshape\footnotesize\color{ChapBlue}}
\renewcommand{\cftsubsectionafterpnum}{\cftparfillskip\hspace*{2em}\par}

\renewenvironment{cases}[1]{\begin{dcases}#1}{\end{dcases}}

\newenvironment{thoerem}[2]{\begin{theo}{#1}{}\trivlist\item #2}{\end{theo}}

\newtcbtheorem[number within=section]{prof}{PROOF}{colback=ChapBlue!20,colframe=ChapBlue!150,fonttitle=\bfseries,arc=0mm,leftrule=0mm,toprule=0mm,bottomrule=1mm,rightrule=0mm,breakable}{prof}

\newtcbtheorem[number within=section]{theo}{THEOREM}{enhanced,title=My title,
    attach boxed title to top center=
        {yshift=-0.25mm-\tcboxedtitleheight/2,yshifttext=2mm-\tcboxedtitleheight/2},
    boxed title style={boxrule=0.5mm,
            frame code={ \path[tcb fill frame] ([xshift=-4mm]frame.west)
                    -- (frame.north west) -- (frame.north east) -- ([xshift=4mm]frame.east)
                    -- (frame.south east) -- (frame.south west) -- cycle; },
            interior code={ \path[tcb fill interior] ([xshift=-2mm]interior.west)
                    -- (interior.north west) -- (interior.north east)
                    -- ([xshift=2mm]interior.east) -- (interior.south east) -- (interior.south west)
                    -- cycle;}
        }, colback = ChapBlue!10!white, colframe = ChapBlue!50!black, sharp corners, colbacktitle = ChapBlue,fonttitle=\bfseries,coltitle=white}{theo}


\renewenvironment{quote}[2]{\begin{tcolorbox}[empty,
            coltitle=ChapBlue!75!black,fonttitle=\bfseries,
            borderline south={0.5mm}{0pt}{ChapBlue!50!white},
            title = {#1},
            titlerule style={ChapBlue,
                    arrows = {|-|}},
            leftrule = 0pt,
            rightrule = 0pt, breakable,
            colupper = darkgray
            ]\slshape\small\color{darkgray} #2}{\end{tcolorbox}}
\newfontface\ebgaramond{EB Garamond}
\usepackage[top=1.25in,bottom=1.3in,outer=1.3in,inner=1.3in]{geometry}
\newsavebox{\ChpNumBox}
\definecolor{ChapBlue}{rgb}{0.00,0.65,0.65}
\setmonofont{Iosevka}
\setCJKmainfont{Noto Serif CJK SC}[ItalicFont = FZKaiS-Extended]
\setCJKsansfont{Noto Sans CJK SC}[ItalicFont = FZKaiS-Extended]


\renewenvironment{proof}[2]{\begin{prof}{#1}{}\pushQED{\qed}\normalfont\topsep6\p@\@plus6\p@\relax\trivlist\item #2}{\popQED\endtrivlist\@endpefalse\end{prof}}
\providecommand{\xlongequal}[2][]{\ext@arrow 0099{\arrowfill@===}{#1}{#2}}
% \titleformat{\section}[display]
% {\vspace*{-3ex}}{\filright
%     \begin{tblr}{colspec = {c}, rows = {abovesep = 3pt,belowsep = -1pt}}
%         \SetCell{bg = ChapBlue!50!white, fg = ChapBlue!80!black}{\bfseries\LARGE\arabic{chapter}.\arabic{section}} \\
%         \vspace*{-2ex}\!\scshape\textit{Section}\!
%     \end{tblr}
%     \setcounter{subsection}{0}
%     \color{ChapBlue!70!black}\special{pdf:literal direct 2 Tr 0.5 w }\Large\(\longmapsto\kern-0.8em{\color{white}\special{pdf:literal direct 2 Tr 0.7 w }∎}\special{pdf:literal direct 2 Tr 0.5 w }\)\kern-1.5em\titlerule*[0.1pc]{\(-\)}\(\!\!\!\longrightarrow\)\special{pdf:literal direct 0 Tr 0 w }}{1em}{\vspace*{-5ex}\Large\bfseries\rightline}[\vspace{-3ex}]

\titleformat{\section}[hang]
{\vspace*{-3ex}}{\begin{tblr}{colspec = {c}, rows = {abovesep = 3pt,belowsep = -1pt}, row{2} = {abovesep = -5pt},baseline = T}
            \SetCell{bg = ChapBlue!50!white, fg = ChapBlue!80!black}{\bfseries\LARGE\arabic{chapter}.\arabic{section}} \\
            \makebox[0pt][c]{\scshape\textit{Section}}
        \end{tblr}}{1em}{\color{ChapBlue!70!black}\special{pdf:literal direct 2 Tr 0.5 w}\LARGE\(\longmapsto\kern-0.8em{\color{white}\special{pdf:literal direct 2 Tr 0.9 w}∎}\special{pdf:literal direct 2 Tr 0.5 w}\)\kern-1.5em\titlerule*[0.1pc]{\(-\)}\(\!\!\!\longrightarrow\)\special{pdf:literal direct 0 Tr 0 w}~~\color{black}\Large\bfseries}[]

\newcommand{\abstractIN}{}

\newcommand{\newchap}[3]{
    \renewcommand{\abstractIN}{#3}
    \chapter[#1]{#2}\mbox{}\par
    \thispagestyle{forchapter}
    \setcounter{subsection}{0}
    \enlargethispage{-0.7in}
}



\titleformat{\subsection}[display]
{\vspace*{-1ex}}{}{}{\colorbox{ChapBlue!20!white}{\color{ChapBlue!90!black}\vspace{-3ex}\stepcounter{subsection}\large\bfseries\bfseries\thechapter.\arabic{section}.\arabic{subsection}}\hspace*{1em}\bfseries}


\makepagestyle{tocf}
\makeevenfoot{tocf}{}{\vskip5ex\colorbox{white}{\color{ChapBlue}~\bfseries\Roman{page}~}}{}
\makeoddfoot{tocf}{}{\vskip5ex\colorbox{white}{\color{ChapBlue}~\bfseries\Roman{page}~}}{}
\makeevenhead{tocf}{}{\ThisTileWallPaper{\paperwidth}{\paperheight}{tocOUT.pdf}}{}
\makeoddhead{tocf}{}{\ThisTileWallPaper{\paperwidth}{\paperheight}{tocIN.pdf}}{}
\makefootrule{tocf}{\textwidth}{2pt}{-9.6ex}
\makeheadfootruleprefix{tocf}{}{\color{ChapBlue}}

\makepagestyle{toc}
\makeheadrule{toc}{\textwidth}{0.4pt}
\makeevenfoot{toc}{}{\vskip5ex\colorbox{white}{\color{ChapBlue}~\bfseries\Roman{page}~}}{}
\makeoddfoot{toc}{}{\vskip5ex\colorbox{white}{\color{ChapBlue}~\bfseries\Roman{page}~}}{}
\makeoddhead{toc}{\ThisTileWallPaper{\paperwidth}{\paperheight}{tocIN.pdf}}{}{\scshape\sffamily 目录 \~{} Contents}
\makeevenhead{toc}{\scshape\sffamily 目录 \~{} Contents}{\ThisTileWallPaper{\paperwidth}{\paperheight}{tocOUT.pdf}}{}
\makefootrule{toc}{\textwidth}{2pt}{-9.6ex}
\makeheadfootruleprefix{toc}{}{\color{ChapBlue}}

\makepagestyle{fancy}
\makeevenfoot{fancy}{\vskip-8em\hspace{-2.5em}\rotatebox[origin=l]{90}{{\LARGE\ebgaramond ☙}\enspace{\bfseries\Roman{page}}\enspace{\LARGE\ebgaramond ❧}}}{}{}
\makeoddfoot{fancy}{}{}{\vskip-8ex\rotatebox[origin=r]{-90}{{\LARGE\ebgaramond ☙}\enspace{\bfseries\Roman{page}}\enspace{\LARGE\ebgaramond ❧}}\kern-2.5em}
\makeevenhead{fancy}{}{\leftmark}{}
\makeoddhead{fancy}{}{\rightmark}{}
\makeheadrule{fancy}{\textwidth}{0.4pt}

\makepagestyle{forchapter}
\makeevenfoot{forchapter}{\ThisTileWallPaper{\paperwidth}{\paperheight}{coverIN.pdf}}{\vskip10pt\color{white}\LARGE\textbf{\arabic{page}}}{}
\makeoddfoot{forchapter}{\ThisTileWallPaper{\paperwidth}{\paperheight}{coverIN.pdf}}{\vskip10pt\color{white}\LARGE\textbf{\arabic{page}}}{}



\newcommand*{\thickhrulefill}{%
    \leavevmode\leaders\hrule height 1\p@ \hfill \kern \z@}
\makechapterstyle{BlueBox}{%
    \setlength{\beforechapskip}{10pt}
    \setlength{\midchapskip}{26pt}
    \setlength{\afterchapskip}{20pt}
    \renewcommand{\printchaptername}{}
    \renewcommand{\chapternamenum}{}
    \def\chapternorthwest{}
    \def\chaptertitleIN{}
    \renewcommand{\printchapternum}{}
    \renewcommand{\printchaptertitle}[1]{\linespread{1}
        \noindent\begin{tblr}{width = \textwidth + 5em , colspec = {X[-1, c]X[h,l]}}
            \makebox[0pt][c]{\scshape Chapter} \strut                            & {\addtolength{\leftskip}{4ex} \thickhrulefill  \\ \Huge\bfseries ##1\strut\vspace*{-10em}} \\
            \SetCell{bg=ChapBlue,fg=white,h,c}{\Huge\bfseries \thechapter\strut} & \SetCell{m}{\qquad\par\vspace*{2em}\TblrAlignBoth\addtolength{\leftskip}{2em}\setlength{\parindent}{2em}\slshape\linespread{1.3}\small\abstractIN \setlength{\parindent}{2em}} \\
        \end{tblr}
    }}

\makeatother
\makechapterstyle{toc}{%
    \setlength{\beforechapskip}{10pt}
    \setlength{\midchapskip}{26pt}
    \setlength{\afterchapskip}{20pt}
    \renewcommand{\printchaptername}{}
    \renewcommand{\chapternamenum}{}
    \def\chapternorthwest{}
    \def\chaptertitleIN{}
    \renewcommand{\printchapternum}{}
    \renewcommand{\printchaptertitle}[1]{
        \noindent\begin{tblr}{width = \textwidth, colspec = {X[-1, c]X[h,r]}}
            \makebox[0pt][c]{\scshape Chapter} \strut                                                                  & {\qquad \thickhrulefill \\ \qquad\Huge\bfseries ##1\strut\vspace*{-10em}} \\
            \SetCell{bg=ChapBlue,fg=white,h,c}{\rotatebox[origin=c]{90}{\color{white}\Huge\bfseries ~Contents~}\strut} &                         \\
        \end{tblr}
    }}


\titleformat{\paragraph}[runin]{\bfseries\color{ChapBlue}}{}{}{{\special{pdf:literal direct 2 Tr 0.3 w }\color{ChapBlue}\ebgaramond\char"261E}\hspace*{0.5em}\special{pdf:literal direct 0 Tr 0 w }}[]

\renewcommand{\contentsname}{目录\\ \LARGE Contents}

\AtBeginDocument{\addtocontents{toc}{\protect\thispagestyle{tocf}}}

   
\setmathfont{NewCMMath-Regular.otf}
\setmathfont[range = {"1D5A0-"1D63B, "1D5D4-"1D66F}]{Garamond-Math.otf}
\DeclareMathOperator{\id}{id}

\def\Obj{\operatorname{Obj}}
\def\Hom{\operatorname{Hom}}
\def\End{\operatorname{End}}
\def\Aut{\operatorname{Aut}}
\def\SET{\symsf{SET}}
\def\st{\text{ \underline{st} }}
\catcode`\。 = 13
\def。{.}
\setlist[enumerate]{label*=\arabic*.}
\everymath{\displaystyle}
\begin{document}

\setlength{\lineskip}{5pt}
\setlength{\lineskiplimit}{2.5pt}
\setlength{\parskip}{0.5em}

\setlist[itemize,2]{label = \(\circ\)}
\chapterstyle{toc}
\thispagestyle{empty}


\clearpage

\pagestyle{toc}\newgeometry{outer = 8cm, bottom = 6cm, top = 5cm, inner = 3cm}
\frontmatter
\tableofcontents*\restoregeometry
\clearpage
\linespread{1.3}

\pagestyle{fancy}

\mainmatter
\chapterstyle{BlueBox}
\setcounter{page}{1}

\newchap{Preliminaries}{Preliminaries: \\
Set theory \& categroies}{}
\section{Naive set theory}


\subsection{集合的运算}

\begin{itemize}
    \item \(\cup:\) union;
    \item \(\cap:\) intersection;
    \item \(\backslash:\) difference;
    \item \(\amalg :\) disjoint union;
    \item \(\times :\) set product;
\end{itemize}

\subsection{disjoint union}

\begin{itemize}
    \item \(S \amalg T\):得到 \(S\) 与 \(T\) 的拷贝 \(S'\) 与 \(T'\),且 \(S' \cap T' = \varnothing\),则 \(S' \cup T' = S\amalg T\)。
    \item 其中一种依赖于 set product 的实现:
          \[
              \begin{cases}
                  S' \coloneqq \left\{ 0 \right\} \times S, \\
                  T' \coloneqq \left\{ 1 \right\} \times T.
              \end{cases}
              \]
\end{itemize}

\subsection{set product}

\[
    S \times T\coloneqq \left\{ \left\{ \left\{ s \right\} ,\left\{ s,t \right\}  \right\} : s\in S\land t\in T \right\}
    .\]

将 \(\left\{ \left\{ s \right\} ,\left\{ s,t \right\}  \right\}\) 写作 \((s,t)\),称为 pair。

\subsection{等价关系}

\begin{itemize}
    \item 若 \(\symcal{R} \) 是二元关系,则 \(a,b\) 满足关系 \(\symcal{R} \) 写为:
          \[
              a\mathop{\symcal{R}}b
              .\]
    \item \textbf{等价关系}:若关系 \(\sim\) 定义在集合 \(S\)上满足:
          \begin{itemize}
              \item reflexivity: \((\forall a\in S) a \sim a\).
              \item symmetry: \((\forall a\in S)(\forall b\in S) a\sim b \implies b \sim a\).
              \item transitivity: \((\forall a\in S)(\forall b\in S)(\forall c\in S) a\sim b \land b\sim c \implies a\sim c\).
          \end{itemize}
          则称 \(\sim\) 是在集合 \(S\) 上的等价关系。
\end{itemize}

\subsection{分划与等价类 (partition \& equivalence class)}

\begin{itemize}
    \item \textbf{分划}是一个集合的集合,满足:
          \[
              \begin{cases}
                  (\forall a\in P) (\forall b \in P) a \cap b = \varnothing, \\
                  \bigcup_{a \in P} a = S.
              \end{cases}
              \]
          则称 \(P\) 是 \(S\) 的分划。
    \item 等价类:
          \[
              [a]_{\sim} \coloneqq  \left\{ x \in S : x \in a \right\}
              .\]
          称此为在 \(S\) 上 \(a\) 的等价类,由于等价类两两不交,且具有自反性,则 \(S\) 上某等价关系得到的所有等价类组成的集合是 \(S\) 的分划 \(\symcal P_{\sim}\)。
\end{itemize}

\subsection{集合商 (set quotient)}
集合 \(S\) 与等价关系 \(\sim\) 的商定义为:
\[
    S /\mathord \sim \coloneqq\symcal{P} _{\sim}
    .\]

即 \(a,b\in S\) 等价 \(\iff\) 商到同一个元素。

\begin{quote}{一个集合商的例子}
    定义 \(\symbb{Z} \) 上的等价关系 \(\sim\) : \(a\sim b \iff \frac{a - b}{2} \in\symbb{Z}\),则:
    \[
        \symbb{Z} /\mathord \sim = \left\{ [0]_{\sim} , [1]_{\sim} \right\}
        .\]
\end{quote}




\section{Functions between sets}


\subsection{函数}
\begin{itemize}
    \item 函数的 Graph:
          \[
              \Gamma _f \coloneqq \left\{ (a,b) \in A \times B : b = f(a) \right\}
              .\]
          且满足 \((\forall a\in A)(\exists !b\in B) (a,b) \in \Gamma_f \),即 \((\forall a\in A)(\exists !b\in B) f(a) = b\)。
    \item 函数的图的表示:

          \[
              \begin{cases}
                  A \xrightarrow{f} B, \\
                  a \mapsto f(a).
              \end{cases}
              \]

\end{itemize}

\subsection{Indentity function(id)}

在集合 \(A\) 上有:
\[
    \id_A : A\to A , (\forall a\in A)\id_A(a) = a
    .\]
\subsection{函数的 image}

若 \(S\subset A, f : A \to B\),则:

\[
    f(S) \coloneqq \left\{ b\in B : (\exists a \in S) f(a) = b  \right\}
    .\]

则 \(f(A)\) 就是函数的image,记作 \(\operatorname{im}f\)

\subsection{函数的 restriction}

记 \(S\subset A\),则:
\[
    f|_S : S \to A, (\forall s\in S) f|_S(s) = f(s)
    .\]

\subsection{函数的复合 (composition)}

\begin{itemize}
    \item 若 \(f:A\to B, g: B\to C\),则 \(g\circ f: A\to C, (\forall a\in A) g\circ f(a) \coloneqq g\left( f(a) \right) \)。
\end{itemize}

\[
    \begin{tikzcd}
        A \arrow[rd, "g \circ f"]\arrow[r, "f"]  &   B\arrow[d, "g"]\\
        {}  &   C
    \end{tikzcd}
    .\]

{\slshape 此时称图是交换(commutative)的,因为图描述的所有从 \(A\) 到 \(C\) 的通路都会送 \(A\) 中的任意一个元素到相同的结果。 }

函数的复合满足结合律:
\[
    \begin{tikzcd}
        A\arrow[r, "f"]\arrow[rr, "g\circ f", bend left = -40]   &   B\arrow[r, "g"]\arrow[rr, "g\circ f", bend left = 40]    &   C\arrow[r, "h"]   &   D
    \end{tikzcd}
    .\]
即 \(h \circ (g \circ f) = (h\circ g) \circ f\)。

\subsection{单射、全射、双射(injections, surjections, bijections)}

\begin{itemize}
    \item 单射 (injections, \textbf{inj})

          \(f: A\to B\) 是单的若:
          \[
              (\forall a'\in A)(\forall a''\in A) a' \neq a'' \implies f\left( a' \right) \neq f(a'')
              .\]
          实际上就是 \((\forall a'\in A)(\forall a''\in A) f\left( a' \right) = f\left( a'' \right) \implies a'=a''\)。
          一般用箭头 \(f: A ↪ B \) 表示。

    \item 全射 (surjections, \textbf{surj})
          \(f : A\to B\) 是满的若:
          \[
              (\forall b\in B)(\exists a\in A) f(a) = b
              .\]
          此时 \(\operatorname{im}f = B\)。
          一般用箭头 \(f: A ↠ B\) 表示。
    \item 双射 (bijections, \textbf{bij})
          \(f\) 是双的当且仅当 \(f\) 又单又满,一般用箭头 \(f: A\stackrel{\sim}{\to }B\) 表示。
          \begin{itemize}
              \item 若 \(\exists f: A \stackrel{\sim}{\to } B\),则记此时 \(A\cong B\),若其中一个集合元素数量有限,则另一个也有限且两个集合元素数量相等。
              \item 集合 \(A\) 中的元素数量写作 \(|A|\);幂集写作 \(2^A\)。
          \end{itemize}
\end{itemize}

\subsection{单射、全射、双射的性质}
\begin{itemize}
    \item 双射有逆 (inverse):
          \begin{theo}{双射有逆}
              定义函数 \(f: A\stackrel{\sim}{\to }B\),定义 \(g: B \to 2^A, (\forall b\in B)g(b) = \left\{ a : f(a) = b \right\} \),则由于 \(f\) 是单的,则 \((\forall a',a''\in A)f(a') = f(a'') \implies a' = a'\),故 \((\forall b\in B ) |g(b)| = 1\)。故可以定义 \(g': B \to A, (\forall b\in B) g(b) = a, \text{ \underline{st} } f(a) = b\),且是良定义的。

              此时 \(g'\circ f = \id_A, f\circ g' = \id_B\),此称 \(g'\) 为 \(f\) 的逆,记为 \(f^{-1} \)。
          \end{theo}

          双射的逆唯一:
          \begin{proof}{双射的逆唯一}
              定义 \(f : A \stackrel{\sim}{\to} B\) 的逆 \(g,g'\),由于 \(f\circ\id_A = f = \id_B\circ f\),因此:
              \[
                  g = g\circ \id_B = g\circ (f \circ g') = (g \circ f) \circ g' = \id_A \circ g' = g'
                  .\]
              故唯一。
          \end{proof}
    \item 左逆与右逆 (Linv \& Rinv)
          若 \(f: A\to B, g\circ f = \id_A\),则称 \(g\) 是 \(f\) 的左逆,同理有右逆。

          如果 \(A \neq \varnothing, f:A\to B\):
          \begin{itemize}
              \item \(f\) 有 Linv \(\iff f\) 是单的。
                    \begin{proof}{}
                        \begin{enumerate}
                            \item (\(\implies \)) 若 \(f\) 有左逆,设为 \(f^{-1} \),则:
                                  \[
                                      (\forall a,b\in A\land a\neq b)f^{-1} (f(a)) = \id_A(a) = a \neq b = f^{-1} (f(b))
                                      .\]
                                  若 \((\exists a,b\in A)f(a) = f(b)\),则与上式矛盾,故 \(f\) 是单的。
                            \item \(\impliedby \) 若 \(f\) 是单的,有双射有逆那部分的讨论知道 \((\forall b\in\operatorname{im}f)\exists !a \in A \text{ \underline{st} } f(a) = b\),故定义:
                                  \[
                                      g(b)\coloneqq \begin{cases}
                                          a, & \text{if }(\exists a\in A)b = f(a),    \\
                                          S, & \text{if } b \notin \operatorname{im}f.
                                      \end{cases}
                                  \]
                                  则 \(g\) 满足 \(g\circ f = \id_A\)。
                        \end{enumerate}
                    \end{proof}
              \item \(f\) 有 Rinv \(\iff f\) 是全的。
                    \begin{proof}
                        只证明 \(\impliedby \) 的部分,由于 \(f\) 是满的,定义:
                        \[
                            g:B\to 2^A, b\mapsto  \left\{ a\in A : f(a) = b\right\}
                            .\]
                        则 \((\forall b\in B)g(b) \neq \varnothing\),定义:
                        \[
                            h:B\to A, b \mapsto a \text{ \underline{st} } a   \in g(b)
                            .\]
                        这样定义的 \(h\) 可能有很多种,但都满足其是 \(f\) 的右逆:
                        \[
                            (\forall b\in B)h(b) \in g(b) \implies f\circ h(b) = \id_B
                            .\]
                    \end{proof}
              \item 若\(f\)同时有左逆和右逆,则两个逆相同。
          \end{itemize}
\end{itemize}
\begin{quote}{一些单射、全射的例子}
    \begin{itemize}
        \item 投影
              \[
                  \begin{tikzcd}
                      {}  &   A\times B\arrow[rd, "\pi_B", two heads]\arrow[ld, "\pi_A"', two heads]   &   {}  \\
                      A   &   {}          &   B   \\
                  \end{tikzcd}
              \]
              其中 \(\pi \) 是投影 (projection) 映射:
              \[
                  \begin{aligned}
                      \pi _A((a,b)) \coloneqq a , \\
                      \pi _B((a,b))\coloneqq b.
                  \end{aligned}
                  .\]
              是全射。

        \item 与不交并的映射
              \[
                  \begin{tikzcd}
                      A\arrow[rd, "f_B",hook]   &   {}          &   B\arrow[ld, "f_A"', hook']   \\
                      {}  &   A\amalg  B   &   {}  \\
                  \end{tikzcd}
              \]
              若将 \(A\amalg B\) 表示为 \(A'\cup B'\),其中 \(A'\overset{F_A}{\cong} A, B'\overset{F_B}\cong B\),则 \((\forall a\in A)f_A(a) \coloneqq F_A(a) \in A\amalg B\)。
        \item 商
              \[
                  \begin{tikzcd}
                      A\arrow[r, "f", two heads]   &   A /\mathord \sim
                  \end{tikzcd}
              \]
        \item 函数的标准分解 (Canonical decomposition)
              对函数 \(f:A\to B\),在 \(A\) 上建立等价关系:
              \[
                  a\sim b \iff f(a) = f(b)
                  .\]
              则函数可以分解为:
              \[
                  \begin{tikzcd}
                      A\arrow[r, two heads]\arrow[rrr, bend left = 40, "f"]   &   A /\mathord \sim \arrow[r, "\sim", "\tilde f"']     &   \operatorname{im}f\arrow[r, hook]  &   B
                  \end{tikzcd}
              \]
              其中 \(\tilde f:A / \mathord\sim \to \operatorname{im} f, \tilde f([a]_{\sim}) \coloneqq f(a)  \),不难验证这是良定义的,现在证明 \(\tilde f\) 是双射:
              \begin{proof}{}
                  只需证明 \(f\) 既是单射也是全射就可以了:
                  \begin{enumerate}
                      \item \textbf{inj}:
                            \[
                                \tilde f\left( [a]_{\sim} \right)  = \tilde f\left( [b]_{\sim} \right) \implies f(a) = f(b) \implies a\sim b \implies [a]_{\sim} = [b]_{\sim}
                                .\]
                      \item \textbf{surj}:
                            \[
                                (\forall b\in \operatorname{im}f)\exists a\in A \text{ \underline{st} }f(a) = b \implies \tilde f\left( [a]_{\sim} \right) = b
                                .\]
                  \end{enumerate}
              \end{proof}
    \end{itemize}
\end{quote}



\section{范畴 (Categories)}



一个范畴 \(\symsf C\)包括:
\begin{itemize}
    \item 一个类 \(\Obj(\symsf C)\),包括了对象 (object)。
    \item 对任意两个对象 \(A,B\) 存在一个集合记为 \(\Hom_{\symsf C}(A,B)\) 包含了从 \(A\) 到 \(B\) 的全部态射 (morphisms),态射和 \(\Hom\) 满足以下特点:
          \begin{itemize}
              \item[\textbf{幺元的存在性}] \(\forall A\in \symsf C,\exists 1_A \in \Hom_{\symsf C}(A,A) \eqcolon \End_{\symsf C}(A)\),称为 \(A\) 的 indentity。
              \item[\textbf{态射复合的存在性}] 若 \(\exists f\in \Hom_{\symsf C}(A,B), g\in \Hom_{\symsf C}(A,B)\),则存在 \(f,g\) 决定的态射 \(gf\in \Hom_{\symsf C}(A,C)\),由于 \(\Hom\) 是集合,因此存在函数:
                  \[
                      \Hom_{\symsf C}(A,B) \times \Hom_{\symsf C}(B,C) \to \Hom_{\symsf C}(A,C)
                      .\]
              \item[\textbf{态射复合的结合性}] 若 \(f\in \Hom_{\symsf C}(A,B), g\in \Hom_{\symsf C}(B,C), h\in \Hom_{\symsf C}(C,D)\),则
                  \[
                      (hg)f = h(gf)
                      .\]
                  这一性质导致态射图可交换。
              \item [\textbf{幺元律}]\[
                        \forall f\in \Hom_{\symsf C}(A,B),f 1_A = 1_Bf=f
                        .\]
          \end{itemize}
\end{itemize}

\begin{quote}{一些范畴的小例子}
    \begin{enumerate}[label = \textbf{\arabic*.}]
        \item 对象为集合、态射为集合函数的范畴,记为 \(\SET\):
              \begin{itemize}
                  \item \(\Obj(\SET) \coloneqq \) 一个包含所有集合的类。
                  \item \(\Hom_{\SET}(A,B) \coloneqq B^A\)。
              \end{itemize}
        \item 一个关于二元运算的范畴:若 \(S\) 上的二元运算 \(\sim\) 满足:
              \[
                  (\forall a,b,c\in S)\begin{cases}
                      a\sim a, \\
                      a\sim b\land b\sim c\implies a\sim c .
                  \end{cases}
              \]
              则定义 \(\symsf C_{\sim}\):
              \begin{itemize}
                  \item \(\Obj\left( \symsf C_{\sim} \right) \coloneqq S\)。
                  \item \(\Hom_{\symsf C_{\sim}}(A,B)\coloneqq \begin{cases}
                            (A,B),       & \text{if }A\sim B,  \\
                            \varnothing, & \text{if }A\nsim B.
                        \end{cases}\)
              \end{itemize}
              定义复合为:
              \[
                  \circ _{\symsf C_{\sim}} :((A,B),(B,C))\mapsto (A,C)
                  .\]
              则其为一个范畴。
              \begin{itemize}
                  \item 一个特例,如果认为 \(S=\symbb{Z} \),\(\sim\) 为 \(\leqslant \),则态射图如下:
                        \[
                            \begin{tikzcd}
                                2\arrow[r]\arrow[rd]   &   3\arrow[r]\arrow[d, "1_3"]   &   4\arrow[r]   &   5   \\
                                {}  &   3\arrow[r]\arrow[rru]   &   4\arrow[ru]   &   {}
                            \end{tikzcd}
                        \]
              \end{itemize}
        \item \textbf{由范畴诱导范畴}
              \begin{itemize}
                  \item [\scshape\bfseries Slice CAT]
                        考虑范畴 \(\symsf C\) 中的对象 \(A\),接下来构建 \(\symsf C_A\):
                        \begin{itemize}
                            \item \(\Obj(\symsf C_A) \coloneqq \)  \(\symsf C\) 中所有到 \(A\) 的态射。
                            \item \(\Hom_{\symsf C_A}(f_1,f_2) = \left\{ \sigma :f_1=f_2 \sigma  \right\} \)。
                        \end{itemize}
                        \[
                            \begin{tikzcd}
                                Z_1\arrow[d, "f_1"']\arrow[rd, "\sigma "] &   {}\\
                                A   & \arrow[l,"f_2"']  Z_2 \\
                            \end{tikzcd}
                        \]
                        \(\symsf C_A\) 中态射的复合取自 \(\symsf C\) 中的态射复合。
                  \item [\scshape\bfseries CoSlice CAT]
                        同理,只不过将从到 \(A\) 变成了 \(A\) 到其他对象的态射。
                  \item [\scshape\bfseries Opp CAT]
                        \begin{itemize}
                            \item \(\Obj(\symsf C^{\symup{op}})\coloneqq \Obj(\symsf C)\)。
                            \item \(\Hom_{\symsf C^{\symup{op}}}(A,B)\coloneqq \Hom_{\symsf C}(B,A)\)。
                        \end{itemize}
              \end{itemize}
    \end{enumerate}
\end{quote}


\subsection{态射们}

\begin{itemize}
    \item \textbf{同构}(Isomorphisms)
          若一个态射 \(f \in \Hom_{\symsf C}(A,B)\) 满足:
          \[
              \exists g\in \Hom_{\symsf C}(B,A) \st gf=1_A, fg=1_B
              .\]
          则 \(f\) 是一个同构,此时记 \(g\) 为 \(f^{-1} \),如果 \(f\) 有左逆、右逆,则它们必然相等(唯一)。
          \begin{itemize}
              \item \(1_A\) 是自逆的。
              \item \((fg)^{-1} = g^{-1} f^{-1} \)。
          \end{itemize}
          \begin{quote}{一些关于逆相关范畴的例子}
              \begin{itemize}
                  \item 一个同构都是 indentity 的例子,用 \((\symbb{Z} ,\leqslant )\)定义的范畴。
                  \item 一个每个态射都是同构的例子,用 \((\symbb{Z} ,=)\) 定义的范畴。这种性质的范畴被称为广群 (Groupoids)。
              \end{itemize}
          \end{quote}
    \item 自同构 (Automorphisms)
          就是属于 \(\End\) 的同构,所有 \(A\) 的自同构组成的集合称为 \(\Aut_{\symsf C}(A)\)
          \begin{itemize}
              \item \(f,g \in\Aut_{\symsf C}(A) \implies fg\in\Aut_{\symsf C}(A)\)。
              \item \(f\in \Aut_{\symsf C}(A) \implies f^{-1} \in\Aut_{\symsf C}(A)\)。
          \end{itemize}

          \(\Aut\) 是一个\textbf{群} (Group)。
    \item 单态射 (Monomophisms, \textbf{Monic})

          即满足左消去律的态射:

          \begin{gather*}
              \forall Z\in \Obj(\symsf C),\forall a,b\in\Hom_{\symsf C}(Z,A),f:A\to B\\
              f \text{ is a monic} \iff \left( f\circ a=f\circ b \implies a=b \right)
          \end{gather*}
    \item 全态射 (Epimorphisms, \textbf{Epic})

          满足右消去律的态射:

          \begin{gather*}
            \forall Z\in \Obj(\symsf C),\forall a,b\in\Hom_{\symsf C}(B,Z) ,f:A\to B\\
            f \text{ is an epic} \iff \left( a\circ f=b\circ f \implies a=b \right)
        \end{gather*}

\end{itemize}


\end{document}


莉莉白可爱! 